\documentclass[aspectratio=169]{beamer}
\usepackage[T1]{fontenc}
\usepackage[utf8]{inputenc}
\usepackage[ngerman]{babel}
\usepackage{lmodern} % in later versions, the package shall (optionally) use fonts according to the official HAW (PowerPoint) template

% These packages are not necessary for presentations in general,
% only used in this example
\usepackage{tikz}
\usepackage{biblatex}
\bibliography{testPresentation.bib}

% Here we tell beamer to use the HAW theme.
\usetheme{HAW}

\title{Meine Präsentation}
\subtitle{Zum Testen}
\author{Lasse Lüder}
\institute{HAW Hamburg Department Informatik}
\date{\today}
% The logo will be displayed in the bottom right corner. The size should remain unchaged.
% If you do not wish to have the logo on your slides, comment the line.
\logo{\mbox{\includegraphics[height=5mm]{hawLogoTemporary.png}}}

\begin{document}

% For now, using \titlepage in the first frame, does not work.
% Use the following block (unchanged)
{
	\setbeamertemplate{footline}{}
	\setbeamertemplate{sidebar right}{}
	\setbeamercolor{background canvas}{bg=HAW main blue}
	\maketitle
}

\begin{frame}{Inhalt}
\tableofcontents
\end{frame}

% If you want some slides, for example the table of contents, at the beginning
% of every section, use this command
\AtBeginSection[] % Do nothing for \section*
{
	\begin{frame}
		\frametitle{Kapitelübersicht}
		\tableofcontents[currentsection] % the current section is set normal, the other ones are greyed out a bit
	\end{frame}
}

\section{Farben}
\begin{frame}
\frametitle{Farben}
\begin{tikzpicture}
\fill [fill=HAW main blue] (0mm,40mm) rectangle (40mm,50mm);
\node [anchor= west, color=HAW main blue] (main blue) at (40mm,45mm) {HAW Hauptblau (RGB 0/60/160)};
\fill [fill=HAW mean blue] (0mm,20mm) rectangle (40mm,30mm);
\node [anchor= west, color=HAW mean blue] (mean blue) at (40mm,25mm) {HAW Mittelblau (RGB 0/150/210)};
\fill [fill=HAW light blue] (0mm,0mm) rectangle (40mm,10mm);
\node [anchor= west, color=HAW light blue] (light blue) at (40mm,05mm) {HAW Hellblau (RGB 160/190/220)};
\end{tikzpicture}

\end{frame}

\section{Zweites Kapitel}
\subsection{Zweites Kapitel Teil eins}

\begin{frame}
\frametitle{Das zweite Kapitel beginnt...}
Mehr Inhalt%
% With \only (and some other commands) you can uncover the contents of a slide step by step. An important difference between \only and \uncover is that text that is hidden takes no space with \only.
% You will notice that the position of the first part changes when the second part is added.
\only<2-3>{, der sich langsam aufblättert}%
\only<4>{, der sich ändert} % but only with \only you can do things like this
\end{frame}

\begin{frame}
\frametitle{Das zweite Kapitel ist in vollem Gange...}
Mehr Inhalt%
% Now the first part is fixed.
\uncover<2-3>{, der sich langsam aufblättert}%
\uncover<4>{, der sich ändert} % but this does not really work with uncover
\end{frame}

\begin{frame}{Das zweite Kapitel endet...}
Noch mehr Inhalt...
\pause

... Schluss!
\end{frame}

\begin{frame}[<+->]{Aufklappen mit unfold und default}
\begin{enumerate}
\item Eins
\item Zwei
\end{enumerate}
\begin{proof}
	qed
\end{proof}
\end{frame}

\subsection{Zweites Kapitel Teil zwei}

\begin{frame}{Aufzählungen}
\begin{itemize}
\item Wichtiger Punkt
\item Noch einer
\end{itemize}
\begin{enumerate}
\item Erstens
\item und zweitens
\end{enumerate}
\end{frame}

\begin{frame}{Fußnoten}
Text \footnote{mit Fußnote}
\end{frame}

\begin{frame}{Fußnoten}
Text \footnote{mit Fußnote} und \footnote{noch einer}

% only one Footnote allowed at this moment (issue #1)
%Noch eine \footnote{weitere Fußnote}
\end{frame}

\begin{frame}{Blocks}
\begin{block}{Blocktitel}
Blockinhalt
\begin{enumerate}
\item Mit einer Enumerierung
\item aus zwei Einträgen.
\end{enumerate}
\end{block}
Nach dem Block
\end{frame}

\begin{frame}{Beweis}
\begin{proof}[Überschrift]
Intelligenter Text
\end{proof}
\end{frame}

\begin{frame}{Beispiel}
\begin{exampleblock}{Beispiel}
Beispiel
\end{exampleblock}
\end{frame}

\begin{frame}{Alarmblock}
\begin{alertblock}{Alarm}
Alarm!
\end{alertblock}
\end{frame}

% If you specifiy allowframebreaks, beamer will spread your content over more
% than one frame and add numbers to the title.
\begin{frame}[allowframebreaks]{Viel Text}
\Large
\begin{enumerate}
\item Wie
\item viele
\item Einträge
\item sind
\item nötig?
\item Mit
\item "`Large"'
\item sind
\item es
\item zehn.
\end{enumerate}
\end{frame}

% Of cource, references can also be used here (although they should be used with care)
\begin{frame}{Referenz}
\autocite{beamer}
\footcite{beamer}
\end{frame}

\begin{frame}{Literatur}
\printbibliography
\end{frame}

% A slide like this is not necessarily a good idea, depending on the type of presentation. It is, however, a classical finish for a ordinary presentations and it is included in the original presentation master.
\closingframe

\end{document}